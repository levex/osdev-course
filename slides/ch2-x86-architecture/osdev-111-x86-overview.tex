\documentclass{beamer}
\usepackage[utf8]{inputenc}
\usepackage[T1]{fontenc}
\usepackage{mathabx}
\usepackage{mathpazo}
\usepackage{eulervm}
\usepackage[mmddyy]{datetime}

\usetheme{Dresden}
\usefonttheme{serif}
\usecolortheme{rose}


\title{OSDEV-111: x86 Architecture overview}
\author{Levente Kurusa}
\institute{lkurusa@acm.org}
\date{Last modified: \today}

\begin{document}

\maketitle

\begin{frame}{Overview}
We will write our operating system for i386 architecture
\begin{itemize}
    \item Multiple "processor modes" ranging from 16 bit to 64 bit.
    \item CISC architecture
    \item Limited set of registers that overlap
\end{itemize}
\end{frame}

\begin{frame}{Real mode}
The mode the CPU starts in.
\begin{itemize}
    \item 16-bit
    \item Ability to use the BIOS to access the devices
    \item Segmented memory layout (more on this later)
    \item Registers available:
    % TODO insert image on register
\end{itemize}
\end{frame}

\begin{frame}{Protected mode}
The mode we will use for our kernel!
\begin{itemize}
    \item 32-bit mode
    \item No BIOS calls (albeit it's possible to circumvent this)
    \item Flat memory layout with protection and paging!
    \item Registers available:
\end{itemize}
\end{frame}

\begin{frame}{Long mode}
We will not use this mode, because it adds extra complexity
\begin{itemize}
    \item 64-bit mode
    \item Flat memory layout with \textbf{required} paging and protection.
    \item Absolutely no BIOS calls
    \item Registers available:
\end{itemize}
\end{frame}

\begin{frame}
It is always possible to switch from one mode to another.
\end{frame}

\end{document}

